https://www.ge.infn.it/~zanghi/metodi/ZULU13.pdf

http://scienze-como.uninsubria.it/complexcomo/download/mat33.pdf

\chapter{Richiami sui numeri complessi}

Il numero complesso permette di fare la radice di un numero negativo, ma deriva dalla necessita di costruire una struttura di campo nell'ambito delle coppie di numeri reali $\mathbb{R}^2$. Ogni elemento di questo insieme è una coppia ordinata di numeri reali $(x,y)$, ora ci chiediamo se sia possibile costruire un campo su questo insieme, ossia dotarlo di una somma e un prodotto. Affinchè questo accada è necessario anche che le operazione di cui dotiamo l'insieme godano di certe proprietà:
\begin{itemize}
    \item $(x,y) + (x',y') = (x+x',y+y')$
\end{itemize}

Elementi base della matematica che useremo.
Forme di rappresentazione dei numeri complessi:
\[
z=a+ib \quad \mbox{Forma algebrica}
\]
\[
z=cos(\theta) + isen(\theta) \quad \mbox{Forma trigonometrica}
\]
\[
z=\rho e^{i\theta} \quad \mbox{Forma di Eulero}
\]
Per ricavare la forma euleriana da quella algebrica, cosa spesso ricorrente e particolarmente utile, usando $\theta\in[0,2\pi]$:
\[
\theta:=Arg(z)=
\begin{cases}
\frac{\pi}{2}				& \text{se $a=0, \, b>0$} \\
\frac{3\pi}{2}				& \text{se $a=0, \, b<0$} \\
\mbox{non definito}			& \text{se $a=0, \, b=0$} \\
arctan(\frac{b}{a})			& \text{se $a>0, \, b\ge 0$} \\
arctan(\frac{b}{a})+2\pi	& \text{se $a>0, \, b<0$} \\
arctan(\frac{b}{a})+\pi		& \text{se $a<0, \, b$ qualsiasi}
\end{cases}
\]
Definiamo le radici n-esime di un numero complesso $z$ nel seguente modo:
\begin{equation}
z^n=a, \quad a=|a|e^{i\alpha} \longrightarrow z_k=|a|^{\frac{1}{n}}e^{i(\frac{\alpha + 2k\pi}{n})}, \quad k=0,...,n-1.
\end{equation}
Così che nel piano complesso si formi sempre un poligono regolare di $n$ lati incentrato nell'origine, da cui si può dimostrare una proprietà operativa molto comoda:
\begin{equation}
\sum_{k=0}^{n-1} z_k=0
\end{equation}
